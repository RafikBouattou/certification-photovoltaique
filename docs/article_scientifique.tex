\documentclass[11pt, a4paper]{article}
\usepackage[utf8]{inputenc}
\usepackage{amsmath}
\usepackage{graphicx}
\usepackage{hyperref}
\usepackage{geometry}
\geometry{a4paper, margin=1in}

\title{\textbf{A Data-Driven Framework for Automated Certification and Performance Analysis of Photovoltaic Sites}}
\author{BOUATTOU, Rafik¹ \\
        ¹Le Labo, Artificial Intelligence & Signal Processing, Algiers, Algeria}
\date{\today}

\begin{document}

\maketitle

\begin{abstract} 
\noindent The increasing scale of photovoltaic (PV) plant deployment necessitates a shift from manual, time-consuming inspection protocols to automated, data-driven solutions for performance certification and fault detection. This paper presents an end-to-end framework for the automated analysis and certification of PV sites based on electrical time-series data. The methodology involves a five-phase pipeline: (1) data preparation and consolidation from disparate sources; (2) calculation of seven key electrical performance indicators (KPIs) with clear mathematical definitions; (3) automated labeling and data balancing using SMOTE to handle class imbalance; (4) predictive modeling using a Random Forest classifier; and (5) automated generation of per-site interactive dashboards for decision support. Applied to a dataset of 72 PV sites, the model achieved a cross-validated accuracy of 97.3% in classifying sites into four compliance categories. The framework successfully identifies underperforming sites and provides granular, interpretable data that pinpoints specific areas of concern, such as DC-side voltage instability. This work demonstrates a scalable and robust solution for proactive operations and maintenance (O&M), enabling asset managers to optimize performance and reduce operational costs.
\end{abstract}

\section{Introduction}

The global transition towards renewable energy has led to an unprecedented expansion of photovoltaic (PV) installations. Ensuring the long-term reliability, safety, and optimal performance of these assets is a primary concern for operators and investors [1]. Traditional methods for site certification and fault diagnosis often rely on periodic manual inspections, which are costly, time-consuming, and may fail to detect intermittent or incipient faults.

The vast amount of data generated by modern PV monitoring systems offers an opportunity to leverage data science and machine learning (ML) for continuous, automated analysis [2]. ML techniques have shown significant promise in detecting various PV system faults, such as partial shading, open-circuit faults, and degradation [3].

This paper proposes a complete, data-driven framework that automates the entire certification process. Our contribution is threefold: (1) An end-to-end, modular pipeline for data processing; (2) The use of a Random Forest model, enhanced by SMOTE, for robust classification; and (3) The automated generation of per-site diagnostic dashboards, bridging the gap between raw data and effective decision-making.

\section{Methodology}

\subsection{Data Acquisition and Preparation}

The initial dataset comprises time-series data from 72 PV sites. The raw data is consolidated, timestamps are aligned to a 1-minute frequency, and missing values are handled via linear interpolation.

\subsection{Feature Engineering: Key Performance Indicators}

Seven KPIs are engineered to characterize site performance. Let \( V_{DC,i} \) be the DC voltage at time \(i\), \(N\) be the total number of samples, and \(\mathbb{I}(\cdot)\) be the indicator function.

\begin{enumerate}
    \item \textbf{DC Voltage Stability (\(\sigma_{V_{DC}}\)):} Standard deviation of the DC voltage.
    \begin{equation}
        \sigma_{V_{DC}} = \sqrt{\frac{1}{N-1} \sum_{i=1}^{N} (V_{DC,i} - \bar{V}_{DC})^2}
    \end{equation}
    
    \item \textbf{AC Voltage Balance (\(V_{bal}\)):} Mean of the maximum deviation between the three AC voltage phases.
    \begin{equation}
        V_{bal} = \frac{1}{N} \sum_{i=1}^{N} \max(|V_{L1,i}-V_{L2,i}|, |V_{L2,i}-V_{L3,i}|, |V_{L3,i}-V_{L1,i}|)
    \end{equation}

    \item \textbf{AC Current Harmony (\(H_{I_{AC}}\)):} Count of current peaks exceeding 1.5 times the mean.
    \begin{equation}
        H_{I_{AC}} = \sum_{i=1}^{N} \mathbb{I}(I_{AC,i} > 1.5 \cdot \bar{I}_{AC})
    \end{equation}

    \item \textbf{AC Frequency Stability (\(H_{f_{AC}}\)):} Count of deviations outside the [49.5, 50.5] Hz grid range.
    \begin{equation}
        H_{f_{AC}} = \sum_{i=1}^{N} \mathbb{I}(|f_{AC,i} - 50| > 0.5)
    \end{equation}

    \item \textbf{Power Factor (PF):} Mean ratio of real power (P) to apparent power (S).
    \begin{equation}
        PF = \frac{1}{N} \sum_{i=1}^{N} \frac{P_{active,i}}{\sqrt{P_{active,i}^2 + P_{reactive,i}^2}}
    \end{equation}

    \item \textbf{Generation-Irradiance Ratio (\(R_{gen/irr}\)):} Mean ratio of generated power to solar irradiance (G).
    \begin{equation}
        R_{gen/irr} = \frac{1}{N} \sum_{i=1}^{N} \frac{P_{gen,i}}{G_i}
    \end{equation}

    \item \textbf{Temporal Variability (\(V_t\)):} Count of abrupt power ramps exceeding 30% of maximum power.
    \begin{equation}
        V_t = \sum_{i=2}^{N} \mathbb{I}(|P_{gen,i} - P_{gen,i-1}| > 0.3 \cdot P_{max})
    \end{equation}
\end{enumerate}

\subsection{Machine Learning Model}

A `RandomForestClassifier` [4] was chosen. The class imbalance was addressed using the SMOTE technique [5]. Model performance was rigorously assessed using a 10-fold stratified cross-validation.

\section{Results}

\subsection{Model Performance}

The model achieved a mean accuracy of \textbf{97.3%}. A summary of performance metrics is presented in Table 1.

\begin{table}[h!]
\centering
\caption{Model Performance Metrics (10-fold Cross-Validation)}
\begin{tabular}{|l|c|}
\hline
\textbf{Metric} & \textbf{Value} \\
\hline
Mean Accuracy & 0.973 \\
Mean Precision (Macro) & 0.978 \\
Mean Recall (Macro) & 0.972 \\
Mean F1-Score (Macro) & 0.971 \\
\hline
\end{tabular}
\end{table}

\subsection{Feature Importance}

The feature importance analysis (Figure 1) revealed that `dc_voltage_stability` and `overall_quality_score` were the most influential factors.

\begin{figure}[h!]
\centering
\includegraphics[width=0.8\textwidth]{feature_importance.png}
\caption{Feature Importance from Random Forest Model.}
\label{fig:feature_importance}
\end{figure}

\section{Discussion}

The results validate the effectiveness of the framework. The high importance of DC voltage stability suggests that many performance issues originate from the PV array itself rather than the grid interface. The framework's ability to flag sites with missing data (e.g., the `NaN` value for the Generation-Irradiance Ratio in our case study) is a critical feature for real-world data quality management.

A primary limitation is the size of the labeled dataset. Future work should focus on acquiring more labeled data and refining metrics, such as the `Ac Current Harmony`, to be more robust against low-power conditions.

\section{Conclusion}

We have demonstrated a robust, end-to-end framework for the automated certification of PV sites. This data-driven approach provides a scalable and efficient alternative to manual inspections, enabling proactive maintenance and performance optimization.

\section{References}

\begin{enumerate}
    \item R. A. Al-Dahidi, et al. (2023). "A Review of Machine Learning-Based Fault Detection and Classification in Photovoltaic Systems." *Energies*.
    \item Y. A. Al-Kuhaili, et al. (2024). "A Review on Faults and Machine Learning-Based Fault Detection in Photovoltaic Systems." *Sustainability*.
    \item M. A. Munim, et al. (2023). "A Review of Machine Learning-Based Fault Detection and Diagnosis for Photovoltaic Systems." *IEEE Access*.
    \item L. Breiman (2001). "Random Forests." *Machine Learning*, 45(1), 5-32.
    \item N. V. Chawla, et al. (2002). "SMOTE: Synthetic Minority Over-sampling Technique." *Journal of Artificial Intelligence Research*, 16, 321-357.
\end{enumerate}

\end{document}
